\chapter[Introdução]{Introdução}

%\textcolor{red}{Esta seção terá no máximo duas páginas para apresentar ao leitor uma breve e atualizada revisão bibliográfica sobre o tema do projeto. A Introdução mostrará ao leitor “como está o mundo atual em relação produto desenvolvido”, ou seja, citará algumas pesquisas com produtos similares, publicadas em Journals ou Teses de Doutorado.}
Quando ideias precisam sair do papel e desafios exigem respostas práticas para os desafios reais do mundo, surge o poder transformador de quem projeta, testa e reinventa. Neste contexto, dominar os resultados de um projeto voltado ao controle de um foguete d'agua  transforma o que poderia parecer uma simples brincadeira de criança em um projeto acadêmico desafiador, que coloca à prova futuros engenheiros. Mais do que lançar objetos aos céus, construímos pontes entre teoria e realidade, nas quais cada teste, simulação e ajuste nos sistemas vira aula prática de física, tecnologia e superação de limites.

Estudar o controle da trajetória de foguetes d’água significa buscar maneiras de lançá-los com mais precisão, atingindo distâncias planejadas. Essa área tem importância não só em atividades educativas, mas também em usos mais amplos, como pesquisas aeroespaciais, sistemas de defesa e até no entretenimento. Este projeto tem como foco desenvolver soluções que tornem possível controlar de maneira mais eficaz o voo e a trajetória desses foguetes. Com base nisso, pode-se definir quais são os objetivos principais que visamos no projeto:

\begin{enumerate}
    \item Alcançar distâncias fixas de 10 metros, 20 metros e 30 metros com uma precisão de
até 0,5 metros
    \item Desenvolver uma plataforma de lançamento que garanta uma distância mínima de
segurança de 5 metros entre o foguete e as pessoas envolvidas no processo.
    \item Reaproveitar o foguete em três lançamentos consecutivos
\end{enumerate} 
A construção e o lançamento de foguetes d’água têm sido explorados por diferentes perspectivas, combinando estudos técnicos e educacionais. Um exemplo disso está no estudo de Apte \cite{Apte2016}, no qual analisam como variáveis como a pressão inicial e o volume de água influenciam a altura máxima atingida por foguetes de garrafa PET movidos apenas por ar comprimido, fornecendo dados experimentais úteis para otimização dos lançamentos. Já na pesquisa de Menezes et al.\ \cite{Menezes2022} demonstra-se o impacto pedagógico desses experimentos em escolas públicas, utilizando foguetes como ferramenta de ensino de ciências e estímulo à participação de meninas na engenharia. Essas abordagens reforçam o valor dos foguetes d’água tanto como recurso didático quanto como objeto de pesquisa aplicada.

Além das contribuições científicas, é fundamental considerar os marcos legais que envolvem a realização de experimentos com lançamentos, como as normas da Agência Nacional de Aviação Civil (ANAC) e da Força Aérea Brasileira (FAB), especialmente no que se refere à segurança em lançamentos de objetos no espaço aéreo. Do ponto de vista de mercado, destaca-se o alto custo de kits comerciais de foguetes educacionais, o que evidencia a relevância de soluções de baixo custo, como a proposta neste projeto. Ainda, observa-se um número reduzido de empresas atuando nesse segmento, o que aponta para uma oportunidade de inovação com potencial impacto social e educacional significativo.

Este projeto se justifica por oferecer uma solução acessível, educativa e ambientalmente responsável, alinhada às necessidades do mundo atual. Ao utilizar materiais recicláveis, como garrafas PET, e aplicar conceitos de física e engenharia no controle de trajetória de foguetes d’água. Além disso, melhora sistemas experimentais já existentes ao incorporar métodos de controle mais precisos e seguros. Dessa forma, o projeto atende a demandas contemporâneas por inovação sustentável, ensino prático e tecnologias de baixo custo, colaborando ativamente com os desafios da educação e do desenvolvimento tecnológico na sociedade atual. Nas seções seguintes, serão detalhadas a metodologia adotada, os resultados esperados e os impactos previstos, evidenciando como esta iniciativa se alinha às tendências atuais na área da engenharia.

%\begin{figure}[htpb]
%\centering
%\includegraphics[width=\textwidth]{figuras/fga.png}
%\caption{\textcolor{red}{Exemplo de figura adicionada em \LaTeX.}}
%\label{fig:exemplo_fig}
%\end{figure}


% \begin{equation}
%     \mathbf{F} = m \mathbf{a}
% \end{equation}

% \begin{equation}
%     x_{1,2} = \frac{-b \pm \sqrt{b^2-4ac}}{2a}
% \end{equation}

% Repare que $b^2-4ac$ pode ser negativo, gerando raízes complexas.

%\textcolor{red}{Além de pesquisas científicas, é essencial que as principais legislações sobre o tema do projeto sejam citadas para atualizar o leitor. Se o grupo ou o professor orientador julgarem relevante, indicadores de mercado devem ser adicionados, como alto custo de produtos similares, baixo número de empresas concorrentes ou número estimado de consumidores finais.}

%\textcolor{red}{A Introdução finalizará com 1 (um) parágrafo de justificativa. Nesse parágrafo, o grupo ressaltará o motivo para determinar que o produto proposto atenda às necessidades atual do mercado/consumidor final ou melhora algum sistema já existente, por exemplo, composto por materiais reciclados.}

%\textcolor{red}{Todas as Tabelas e Figuras devem ser referenciadas ao longo do texto, já que elas são ferramentas para auxiliar no entendimento do mesmo. Use o comando ``\textsf{\textbackslash label\{\}}'' junto a Tabelas e Figuras para referencia-las. A Fig. \ref{fig:exemplo_fig} exemplifica como adicionar imagens ao texto.}

%\textcolor{red}{Para citarem trabalhos, utilizem o comando ``\textsf{\textbackslash cite\{\}}''. Evitem ao máximo o uso de outros comandos, tais como o ``\textsf{\textbackslash citeonline\{\}}''.}

% \chapter*[Introdução]{Introdução}
% \addcontentsline{toc}{chapter}{Introdução}

% Este documento apresenta considerações gerais e preliminares relacionadas 
% à redação de relatórios de Projeto de Graduação da Faculdade UnB Gama 
% (FGA). São abordados os diferentes aspectos sobre a estrutura do trabalho, 
% uso de programas de auxilio a edição, tiragem de cópias, encadernação, etc.

% Este template é uma adaptação do ABNTeX2\footnote{\url{https://github.com/abntex/abntex2/}}.
