\chapter{Termo de Abertura do Projeto}

%\textcolor{red}{Termo de abertura do projeto / Project Charter. Um documento publicado pelo iniciador ou patrocinador do projeto que autoriza formalmente a existência de um projeto e fornece ao gerente do projeto a autoridade para aplicar os recursos organizacionais nas atividades do projeto.}

\section{Dados do projeto}
\begin{description}
    \item [Nome do Projeto:] Controle de trajetória de foguetes d'água 
    \item [Data de abertura:] 16/04/2025
    \item [Código:] 1-A
    \item [Patrocinador:] Universidade de Brasília
    % \item [Responsável:] ???
    \item [Gerente:] Letícia Maria de Souza/231026447/231026447@aluno.unb.br(61)998196388
\end{description}

\section{Objetivos}
%\textcolor{red}{O que a empresa pretende obter com a realização do projeto. Descrever o que se pretende realizar para resolver o problema central ou explorar a oportunidade identificada. Para a correta definição do objetivo siga a regra "SMART":
%\begin{description}
   % \item [\textit{Specific} (específico):] Deve ser redigido de forma clara, concisa e compreensiva;
   % \item [\textit{Measurable} (mensurável):] O objetivo específico deve ser mensurável, ou seja, 
O objetivo principal do presente projeto é realizar uma integração entre diferentes áreas de conhecimento das engenharias, tais quais a Aeroespacial, a Automotiva, a Eletrônica, a de Energia e a de Software. Assim, é possível criar um ambiente de desenvolvimento e de resolução de problemas variado, no qual deve ocorrer de uma comunicação clara e objetiva. Dessa maneira, para alcançar os resultados almejados pelos requsitos do projeto, os estudantes desses diferentes domínios de atuação aprimoram suas habilidades de pesquisa, de transmissão de informações e seu trabalho em grupo.

    Os membros da equipe foram dividos de acordo com as áreas previamente citadas, formando quatro núcleos de trabalho: eletrônica, energia, estruturas e software. Cada área da engenharia estaria responsável por atividades específicas no desenvolvimento do projeto. O núcleo de eletrônica fará o circuito de hardware do foguete, que coletará dados do lançamento e permitirá o acionamento a distância, já o de energia calculará o consumo energético dos sistemas. Já o de estruturas projetará, com cálculos de dimensionamento e desenhos técnicos, o foguete e a base de lançamento, enquanto o núcleo de software será responsável por processar e apresentar os dados coletados.

    Ao final do projeto, para validá-lo, o foguete desenvolvido deverá ser capaz de, após ser lançado da base, alcançar as distâncias de dez, vinte e trinta metros, com precisão de meio metro, utilizando-se da mesma estrutura em todas elas. Além disso, o lançamento deve ser feito a uma distância de cinco metros entre a equipe e a base de lançamento.
    
  %  \item [\textit{Agreed} (acordado):] Deve ser acordado com as partes interessadas, ou seja, as áreas envolvidas na empresa: P\&D, Produção, Comercial, Marketing, Financeira, Jurídica, Manutenção, ambiental, entre outras;
   % \item [\textit{Realistic} (realista):] Deve estar centrado na realidade, no que é possível de ser feito considerando as premissas e restrições existentes, como: orçamento e tempo;
   % \item [\textit{Time Bound} (Limitado no tempo):] Deve ter um prazo determinado para sua finalização.
%\end{description}

%\section{Mercado-alvo}

%\textcolor{red}{Pessoas, empresas, instituições etc. que usufruirão dos produtos, serviços e resultados gerados pelo projeto, cujos requisitos (tópico abaixo) devem atender as suas necessidades. Podem ser internas ou externas à organização, mas, merecem destaque especial, pois, o projeto está sendo feito para atendê-los de forma direta ou indireta.}

%\section{Requisitos}
   % Em todo projeto é necessária a defnição de requisitos que guiem o as etapas do trabalho até atingir os resultados desejados. Dentre os requisitos visados para o projeto estão:
   % \begin{description}
   % \item[] - Reutilização da mesma estrtutura para três lançamentos de distâncias variadas; 
   %\end{description}
    
%\textcolor{red}{Listar os fatos que são essenciais para o consumidor adquirir o produto, exemplo: cor, tamanho, material, tempo de vida-útil, dispositivo de segurança \textbf{x}, entre outros.}

%\section{Justificativa}

%\textcolor{red}{Informar o problema ou a oportunidade (necessidade) que justifica o porquê de o projeto ser realizado. Por exemplo: atende uma demanda específica do consumidor final; supre uma necessidade do mercado comercializador; é um diferencial X para o órgão regulamentador.}

%\section{Indicadores}

%\textcolor{red}{Listar até 10 indicadores que determinam o mercado consumidor do produto desenvolvido: exemplo: 1) n° de alunos da FGA que utilizam ônibus às 18:00; 2) n° de usuários do restaurante universitários, 3) número de idosos classificados como público-alvo no DF e no estado de Goiás, 4) n° de empresas de segurança registradas no DF etc.}
% \section{Aprovações}

% \begin{tabular}{ l l }
%   \textbf{Patrocinador:} & \_\_\_\_\_\_\_\_\_\_\_\_\_\_\_\_\_\_\_\_\_\_\_\_\_\_\_\_\_\_\_\_\_\_ \\
%   & \\
%   \textbf{CEO:} & \_\_\_\_\_\_\_\_\_\_\_\_\_\_\_\_\_\_\_\_\_\_\_\_\_\_\_\_\_\_\_\_\_\_ \\
%   & \\
%   \textbf{Gerente:} & \_\_\_\_\_\_\_\_\_\_\_\_\_\_\_\_\_\_\_\_\_\_\_\_\_\_\_\_\_\_\_\_\_\_ \\
% \end{tabular}

