\chapter{Termo de Abertura do Projeto}
%\textcolor{red}{Termo de abertura do projeto / Project Charter. Um documento publicado pelo iniciador ou patrocinador do projeto que autoriza formalmente a existência de um projeto e fornece ao gerente do projeto a autoridade para aplicar os recursos organizacionais nas atividades do projeto.}
\section{Dados do projeto}
\begin{description}
    \item [Nome do Projeto:] Controle de trajetória de foguetes d'água 
    \item [Data de abertura:] 16/04/2025
    \item [Código:] 1-A
    \item [Patrocinador:] Universidade de Brasília
    % \item [Responsável:] ???
    \item [Gerente:] Letícia Maria de Souza/231026447/231026447@aluno.unb.br/(61)998196388
\end{description}

\section{Objetivos}
%\textcolor{red}{O que a empresa pretende obter com a realização do projeto. Descrever o que se pretende realizar para resolver o problema central ou explorar a oportunidade identificada. Para a correta definição do objetivo siga a regra "SMART":
%\begin{description}
   % \item [\textit{Specific} (específico):] Deve ser redigido de forma clara, concisa e compreensiva;
   % \item [\textit{Measurable} (mensurável):] O objetivo específico deve ser mensurável, ou seja, 
O objetivo principal do presente projeto é realizar uma integração entre diferentes áreas de conhecimento das engenharias, tais quais a Aeroespacial, a Automotiva, a Eletrônica, a de Energia e a de Software. Assim, é possível criar um ambiente de desenvolvimento e de resolução de problemas variado, no qual deve ocorrer de uma comunicação clara e objetiva. Dessa maneira, para alcançar os resultados almejados pelos requsitos do projeto, os estudantes desses diferentes domínios de atuação aprimoram suas habilidades de pesquisa, de transmissão de informações e seu trabalho em grupo.

    Os membros da equipe foram dividos de acordo com as áreas previamente citadas, formando quatro núcleos de trabalho: eletrônica, energia, estruturas e software. Cada área da engenharia estaria responsável por atividades específicas no desenvolvimento do projeto. O núcleo de eletrônica fará o circuito de hardware do foguete, que coletará dados do lançamento e permitirá o acionamento a distância, já o de energia calculará o consumo energético dos sistemas. Já o de estruturas projetará, com cálculos de dimensionamento e desenhos técnicos, o foguete e a base de lançamento, enquanto o núcleo de software será responsável por processar e apresentar os dados coletados.

    Ao final do projeto, para validá-lo, o foguete desenvolvido deverá ser capaz de, após ser lançado da base, alcançar as distâncias de dez, vinte e trinta metros, com precisão de meio metro, utilizando-se da mesma estrutura em todas elas. Além disso, o lançamento deve ser feito a uma distância de cinco metros entre a equipe e a base de lançamento.}
    
  %  \item [\textit{Agreed} (acordado):] Deve ser acordado com as partes interessadas, ou seja, as áreas envolvidas na empresa: P\&D, Produção, Comercial, Marketing, Financeira, Jurídica, Manutenção, ambiental, entre outras;
   % \item [\textit{Realistic} (realista):] Deve estar centrado na realidade, no que é possível de ser feito considerando as premissas e restrições existentes, como: orçamento e tempo;
   % \item [\textit{Time Bound} (Limitado no tempo):] Deve ter um prazo determinado para sua finalização.
%\end{description}


%\section{Mercado-alvo}

\textcolor{black}{O mercado-alvo principal deste projeto são instituições de ensino, como universidades e escolas, que buscam ferramentas educacionais e experimentais de baixo custo para o aprendizado prático em áreas como física, aerodinâmica e sistemas embarcados. Estudantes que se interessam por engenharias e ciências, especialmente aqueles envolvidos em projetos integradores, constituem um segmento direto. Como também, o projeto pode atrair entusiastas de foguetes amadores e pesquisadores que necessitam de plataformas experimentais acessíveis, dado o alto custo dos kits comerciais existentes que são usados para fazer foguetes mais precisos com diferentes custos associados.}

%\section{Requisitos}
    Os requisitos essenciais do produto incluem o desenvolvimento de um foguete d'água reutilizável, capaz de alcançar distâncias fixas de 10m, 20m e 30m com uma margem de precisão de 0,5m. O foguete deve ser acionado eletromecanicamente a uma distância segura de no mínimo 5 metros. A estrutura, construída predominantemente com garrafas PET e PVC, deve suportar três lançamentos consecutivos sem danos. O sistema embarcado deve possuir uma fonte de energia e realizar medições em tempo real de parâmetros como altura, velocidade, ângulo de lançamento e volume de água, com os dados sendo processados e apresentados via software. A base de lançamento deve possuir ângulo fixo, com a variação de alcance controlada pela pressão interna, e incluir um sistema de medição de pressão e vedação eficiente.
   % \begin{description}
   % \item[] - Reutilização da mesma estrtutura para três lançamentos de distâncias variadas; 
   %\end{description}
    
%\textcolor{red}{Listar os fatos que são essenciais para o consumidor adquirir o produto, exemplo: cor, tamanho, material, tempo de vida-útil, dispositivo de segurança \textbf{x}, entre outros.}

%\section{Justificativa}

\textcolor{black}{A realização deste projeto justifica-se pela necessidade de soluções acessíveis e educativas no campo da engenharia experimental. O projeto visa o trabalho em equipe prático e teórico com caráter educacional. O projeto promove a aplicação prática de conceitos de física e engenharia utilizando materiais recicláveis, como garrafas PET, alinhando-se com demandas por inovação sustentável e tecnologias de baixo custo. Além disso, visa aprimorar sistemas experimentais existentes ao incorporar métodos de controle mais precisos e seguros, fomentando o aprendizado multidisciplinar. }

%\section{Indicadores}

\textcolor{black}{Para determinar o mercado consumidor do produto desenvolvido, podem ser considerados os seguintes indicadores: 1) Número de instituições de ensino (níveis fundamental, médio e superior) com programas ou disciplinas que envolvem princípios de física e engenharia aeroespacial. 2) Quantidade de estudantes matriculados em cursos de engenharia e ciências exatas que desenvolvem projetos práticos. 3) Demanda por kits educacionais de baixo custo que demonstrem conceitos científicos complexos. 4) Crescimento do interesse por atividades de ciência, tecnologia, engenharia e matemática (STEM) em feiras de ciências e competições acadêmicas. 5) Número de pesquisas e publicações acadêmicas relacionadas a foguetes de água e projetos educacionais de baixo custo. 6) Volume de buscas online e em comunidades de prática por projetos de foguetes caseiros e experimentais. 7) Disponibilidade de elaboração de projetos de inovação pedagógica e tecnológica em educação. 8) Comparativo de custos entre a solução proposta e os kits comerciais disponíveis. 9) Número de downloads ou acessos a projetos de hardware aberto com finalidade educacional similar. 10) Feedback e taxa de adoção por parte de educadores e alunos em fases de teste ou divulgação do projeto.}
% \section{Aprovações}

% \begin{tabular}{ l l }
%   \textbf{Patrocinador:} & \_\_\_\_\_\_\_\_\_\_\_\_\_\_\_\_\_\_\_\_\_\_\_\_\_\_\_\_\_\_\_\_\_\_ \\
%   & \\
%   \textbf{CEO:} & \_\_\_\_\_\_\_\_\_\_\_\_\_\_\_\_\_\_\_\_\_\_\_\_\_\_\_\_\_\_\_\_\_\_ \\
%   & \\
%   \textbf{Gerente:} & \_\_\_\_\_\_\_\_\_\_\_\_\_\_\_\_\_\_\_\_\_\_\_\_\_\_\_\_\_\_\_\_\_\_ \\
% \end{tabular}

